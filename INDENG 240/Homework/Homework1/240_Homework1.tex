\documentclass{article}
\usepackage{array}
\usepackage{amsfonts}
\usepackage{amsmath}
\usepackage{geometry}
\usepackage{stmaryrd}
\usepackage{xcolor}
\usepackage{systeme}
\geometry{hmargin=2.5cm,vmargin=1.5cm}

\definecolor{Zgris}{rgb}{0.95,0.95,0.95}
\definecolor{line1}{rgb}{0.74,0.35,0.74}
\definecolor{line2}{rgb}{0.50,0.60,0.79}


\newsavebox{\BBbox}
\newenvironment{DDbox}[1]{
\begin{lrbox}{\BBbox}
    \begin{minipage}{\linewidth}}
{\end{minipage}
\end{lrbox}\noindent\colorbox{Zgris}{\usebox{\BBbox}} \\
[.5cm]}


\title{INDENG 240 : Homework 1}
\author{Arnaud Minondo}



\begin{document}


% Use \lstset to make myStyle the global default
\maketitle
\section*{Production Planning Problem}

My model file : pb1.mod
\\
\begin{DDbox}{\linewidth}
    \begin{verbatim}
        param n;
        param order{i in 1..n};

        param costPerTon{i in 1..n};
        param inventoryCondition{i in 1..n};


        param capacity ;
        param invCostPerTon ;
        param startInventory ;
        param prodChangeCost ;


        param matrix{i in 1..n,j in 1..n};


        var production{i in 1..n}>=0;
        var u{j in 1..(n-1)} >= 0;



        minimize cost: (sum{i in 1..n} production[i]*costPerTon[i])
        + prodChangeCost*(sum{j in 1..(n-1)}u[j])
        + invCostPerTon*(3*(production[1]-order[1]+startInventory)+
        2*(production[2]-order[2])+(production[3]- order[3]));

        s.t. realLife{i in 1..n}: startInventory +
         (sum{j in 1..n}(production[j]-order[j])*matrix[i,j]) >= inventoryCondition[i];
        s.t. plantCapacity{i in 1..n} : production[i]<=capacity;
        s.t. absolute1{i in 1..(n-1)}: u[i]>=production[i]-production[i+1];
        s.t. absolute2{i in 1..(n-1)}: u[i]>=production[i+1]-production[i];
    \end{verbatim}
\end{DDbox}
\newpage
My data file : pb1.dat
\\
\begin{DDbox}{\linewidth}
    \begin{verbatim}
        param n := 4;

        param order :=
        1 2400
        2 2000
        3 2700
        4 2500
        ;
        
        param costPerTon :=
        1 7400
        2 7500
        3 7600
        4 7800
        ;

        param inventoryCondition :=
        1 0
        2 0
        3 0
        4 1500
        ;
        param matrix:
        1 2 3 4 :=
        1 1 0 0 0
        2 1 1 0 0
        3 1 1 1 0
        4 1 1 1 1
        ; 
        param capacity := 4000;
        param invCostPerTon := 120;
        param startInventory := 1000;
        param prodChangeCost := 50;
    \end{verbatim}
\end{DDbox}
\\
My run file : pb1.run
\\
\begin{DDbox}{\linewidth}
    \begin{verbatim}
        reset;

        model pb1.mod;

        data pb1.dat;

        option solver cplex;

        solve;

        display production, cost;

        display production, cost > answer_pb1.txt;
    \end{verbatim}
\end{DDbox}
\\
The answer for the problem is : production = (2525,2525,2525,2525) with a cost of 77017500.

\section*{Betting Problem}
My model file : pb2.mod
\\
\begin{DDbox}{\linewidth}
    \begin{verbatim}
        set SCENS;
        set VARS;


        

        param matrix {SCENS, VARS};

        param amount_bet;

        var strategy{VARS} >= 0;
        var w >= 0;

        maximize win: w;


        s.t. budget: sum{i in VARS}strategy[i] <= amount_bet;
        s.t. scen{j in SCENS}: w <= sum{i in VARS}matrix[j, i]*strategy[i];
    \end{verbatim}
\end{DDbox}
\\
My data file : pb2.dat
\\
\begin{DDbox}{\linewidth}
    \begin{verbatim}
        set SCENS = AA ABA ABB BB BAB BAA ;
        set VARS = x1a x2a x3a x1b x2b x3b ya yb;
        
        param amount_bet := 100;
        
        
        param matrix: 
            x1a   x1b   x2a  x2b   x3a   x3b   ya    yb :=
        AA  0.67   -1   0.67 -1     0     0    0.4   -1
        ABA 0.67   -1   -1   1.5   0.67  -1    0.4   -1
        ABB 0.67   -1   -1   1.5   -1    1.5   -1    2.5
        BB  -1    1.5   -1   1.5    0     0    -1    2.5
        BAB -1    1.5   0.67 -1    -1    1.5   -1    2.5
        BAA -1    1.5   0.67 -1    0.67  -1    0.4   -1
        ;
    \end{verbatim}
\end{DDbox}
\\
My run file : pb2.run
\\
\begin{DDbox}{\linewidth}
    \begin{verbatim}
        reset;

        model pb2.mod;

        data pb2.dat;

        option solver cplex;
        solve;

        display strategy, win;

        display strategy, win > pb2_answer.txt;
    \end{verbatim}
\end{DDbox}
\newpage
The optimal solution of this problem is strategy = x1a  17.6 x1b   0 x2a  17.6 x2b   0 x3a  43.8685 x3b   0 ya   0 yb  20.9316
where xia represents the amount bet on 'a' wins the i-th set,
 xib represent the amount bet on 'b' wins the i-th set and
 ya is the amount bet on 'a' wins the match and yb is the amount bet on 'b' wins the match.
You are ensured to win at least \$2.65.
\section*{Regression Problem}
My model file : pb3.mod
\\
\begin{DDbox}{\linewidth}
    \begin{verbatim}
        param n;
        param grades{i in 1..n};

        param degree;

        param hoursWorked{i in 1..n, j in 1..degree};


        var u{i in 1..n};
        var coefficients{i in 1..degree};

        minimize Error :sum{i in 1..n}u[i];

        s.t. absolute1{i in 1..n}: 
        u[i] >= (sum{j in 1..degree}(coefficients[j]*hoursWorked[i,j]))-grades[i];
        s.t. absolute2{i in 1..n}: 
        u[i] >= (-1)*(sum{j in 1..degree}(coefficients[j]*hoursWorked[i,j])-grades[i]);
    \end{verbatim}
\end{DDbox}
\\
My data file : pb3.dat
\\
\begin{DDbox}{\linewidth}
    \begin{verbatim}
        param n := 8;

        #bias introduced in the input
        param hoursWorked :1 2 := 
        1 4 1
        2 9 1
        3 10 1
        4 14 1
        5 4 1
        6 7 1
        7 12 1
        8 3 1;

        param grades := 
        1 75
        2 82
        3 92
        4 100
        5 68
        6 88
        7 95
        8 77;

        param degree := 2;

    \end{verbatim}
\end{DDbox}
\\
\newpage
My run file : pb3.run
\\
\begin{DDbox}{\linewidth}
    \begin{verbatim}
        reset;

        model pb3.mod;

        data pb3.dat;

        option solver cplex;
        solve;

        display coefficients,Error;
        display coefficients,Error>answer_pb3.txt;
    \end{verbatim}
\end{DDbox}
\\
The answer for this problem is coefficients=(2.5,65) with an Error of 24.5.
The prediction for a student having worked only two hours is 2.5*2+65 = 70.
\section*{Solving LP Graphically}
\subsection*{a.}
I draw the conditions on the graph below. After the theorem of linear optimization, the optimal solution is on a border.
 As the function to maximize is $f(x,y)=3x+y$ it is most likely to be optimal for higher x.
 benefit at the first critical point : $\systeme{y=0,2x+y=6}$ so $\systeme{x=3,y=0}$ with $f(x,y)=9$.
\\
We need to check if the neighbours of this point are better or worse than this point :
\\
$\systeme{2x+y=6,x+y=4}$ so $\systeme{x=2,y=2}$, whith a cost function $f(2,2)=8$.
\\
And obviously, (0,0) is not the maximizing point so (3,0) is the optimal solution.
\\
\setlength{\unitlength}{0.75cm}
\begin{picture}(.5,.5)
    \put(0,-10){\vector(1,0){7}}
    \put(0,-10){\vector(0,1){7}}
    \multiput(0, -10)(1, 0){7}{\line(0,-1){0.1}}
    \multiput(0, -10)(0, 1){7}{\line(-1,0){0.1}}
    %axe x
    \put(-0.3,-10.5){0}
    \put(0.9,-10.5){1}
    \put(1.9,-10.5){2}
    \put(2.9,-10.5){3}
    \put(3.9,-10.5){4}
    \put(4.9,-10.5){5}
    \put(5.9,-10.5){6}
    
    %axe y
    \put(-0.4,-9.1){1}
    \put(-0.4,-8.1){2}
    \put(-0.4,-7.1){3}
    \put(-0.4,-6.1){4}
    \put(-0.4,-5.1){5}
    \put(-0.4,-4.1){6}
    
    %plot condition
    \color{line1}
    \put(0,-4){\line(1,-2){3}}
    \color{line2}
    \put(0,-6){\line(1,-1){4}}
    
    %plot legend
    \color{black}
    \put(8.7,-7.5){\line(-1,0){5.4}}
    \put(8.7,-7.5){\line(0,1){3.5}}
    \put(8.7,-4){\line(-1,0){5.4}}
    \put(3.3,-7.5){\line(0,1){3.5}}

    \color{line1}
    \put(3.6,-4.55){\line(1,0){.5}}
    \color{black}
    \put(4.2,-4.7){border1 : $y = 6-2x$}

    \color{line2}
    \put(3.6,-5.35){\line(1,0){.5}}
    \color{black}
    \put(4.2,-5.5){border2 : $y = 4-x$}

    \put(3.6,-6.3){\line(1,1){.5}}
    \put(4.2,-6.2){decision region}

    \color{green}
    \put(3.6,-6.8){\line(1,0){.5}}
    \color{black}
    \put(4.2,-6.9){$f(x,y) = 3$}
    \color{black}
    %plot decision region
    \put(2,-8){\line(-1,-2){1}}
    \put(1,-7){\line(-1,-2){1}}
    \put(0.5,-6.5){\line(-1,-2){.5}}
    \put(1.5,-7.5){\line(-1,-2){1.25}}
    \put(2.5,-9){\line(-1,-2){.5}}
    %plot maximizer
    \color{green}
    \put(-1,-4){\line(1,-3){3}}
\end{picture}
\newpage
\subsection*{b.}
Every borders are the same for this variant of the problem, only the cost function changes : $f(x,y) = x + y$
\\
As $x+y\leq4$ then $max_{x,y}(f(x,y))<=4$ and x=2,y=2 verify this so $x = 2, y = 2$ is one optimal solution but there exist plenty of them, for example : $x=0.5, y =3.5$.
\\
\begin{picture}(.5,.5)
    \put(0,-10){\vector(1,0){7}}
    \put(0,-10){\vector(0,1){7}}
    \multiput(0, -10)(1, 0){7}{\line(0,-1){0.1}}
    \multiput(0, -10)(0, 1){7}{\line(-1,0){0.1}}
    %axe x
    \put(-0.3,-10.5){0}
    \put(0.9,-10.5){1}
    \put(1.9,-10.5){2}
    \put(2.9,-10.5){3}
    \put(3.9,-10.5){4}
    \put(4.9,-10.5){5}
    \put(5.9,-10.5){6}
    
    %axe y
    \put(-0.4,-9.1){1}
    \put(-0.4,-8.1){2}
    \put(-0.4,-7.1){3}
    \put(-0.4,-6.1){4}
    \put(-0.4,-5.1){5}
    \put(-0.4,-4.1){6}
    
    %plot condition
    \color{line1}
    \put(0,-4){\line(1,-2){3}}
    \color{line2}
    \put(0,-6){\line(1,-1){4}}
    
    %plot legend
    \color{black}
    \put(8.7,-7.5){\line(-1,0){5.4}}
    \put(8.7,-7.5){\line(0,1){3.5}}
    \put(8.7,-4){\line(-1,0){5.4}}
    \put(3.3,-7.5){\line(0,1){3.5}}

    \color{line1}
    \put(3.6,-4.55){\line(1,0){.5}}
    \color{black}
    \put(4.2,-4.7){border1 : $y = 6-2x$}

    \color{line2}
    \put(3.6,-5.35){\line(1,0){.5}}
    \color{black}
    \put(4.2,-5.5){border2 : $y = 4-x$}

    \put(3.6,-6.3){\line(1,1){.5}}
    \put(4.2,-6.2){decision region}

    \color{green}
    \put(3.6,-6.8){\line(1,0){.5}}
    \color{black}
    \put(4.2,-6.9){$f(x,y) = 2$}
    \color{black}
    %plot decision region
    \put(2,-8){\line(-1,-2){1}}
    \put(1,-7){\line(-1,-2){1}}
    \put(0.5,-6.5){\line(-1,-2){.5}}
    \put(1.5,-7.5){\line(-1,-2){1.25}}
    \put(2.5,-9){\line(-1,-2){.5}}
    %plot maximizer
    \color{green}
    \put(-1,-7){\line(1,-1){4}}
\end{picture}
\newpage
\subsection*{c.}
The borders condition have changed but there is at the same time : $\systeme{-x + y \leq -1, -x + y \ge 1 }$ which is impossible
\subsection*{d.}
For this problem you can see in the plot, it is unbounded. Let's suppose there exist an optimal solution
 $(x_0,y_0)$ such that $\forall (x,y)\in \mathbb{R}^2, f(x,y) \leq f(x_0,y_0)=x_0+y_0$.
 Define $x_1=x_0 +1,y_1=y_0+1$, $x_1-y_1 = x_0-y_0 $ so $(x_1,y_1)$ is a valid couple as it verifies $-1\leq x_1-y_1\leq1$.
 Moreover : $f(x_1,y_1) = x_0+y_0+2 = f(x_0,y_0) +2 > f(x_0,y_0)$ which contradicts the fact that $(x_0, y_0)$ is the optimal couple.
 So there can't exist a maximum for this function under these limit conditions.
 \\
\begin{picture}(.5,.5)
    \put(0,-10){\vector(1,0){7}}
    \put(0,-10){\vector(0,1){7}}
    \multiput(0, -10)(1, 0){7}{\line(0,-1){0.1}}
    \multiput(0, -10)(0, 1){7}{\line(-1,0){0.1}}
    %axe x
    \put(-0.3,-10.5){0}
    \put(0.9,-10.5){1}
    \put(1.9,-10.5){2}
    \put(2.9,-10.5){3}
    \put(3.9,-10.5){4}
    \put(4.9,-10.5){5}
    \put(5.9,-10.5){6}
    
    %axe y
    \put(-0.4,-9.1){1}
    \put(-0.4,-8.1){2}
    \put(-0.4,-7.1){3}
    \put(-0.4,-6.1){4}
    \put(-0.4,-5.1){5}
    \put(-0.4,-4.1){6}
    
    %plot borders
    \color{line1}
    \put(-1,-10){\line(1,1){4}}
    \color{line2}
    \put(0,-11){\line(1,1){3}}
    
    %plot legend
    \color{black}
    \put(8.7,-7.5){\line(-1,0){5.4}}
    \put(8.7,-7.5){\line(0,1){3.5}}
    \put(8.7,-4){\line(-1,0){5.4}}
    \put(3.3,-7.5){\line(0,1){3.5}}

    \color{line1}
    \put(3.6,-4.55){\line(1,0){.5}}
    \color{black}
    \put(4.2,-4.7){border1 : $y = 6-2x$}

    \color{line2}
    \put(3.6,-5.35){\line(1,0){.5}}
    \color{black}
    \put(4.2,-5.5){border2 : $y = 4-x$}

    \color{red}
    \put(3.6,-6.3){\vector(1,1){.5}}
    \color{black}
    \put(4.2,-6.2){max direction}

    \color{green}
    \put(3.6,-6.8){\line(1,0){.5}}
    \color{black}
    \put(4.2,-6.9){$f(x,y) = 1$}

    \color{red}
    \put(0.5,-9.5){\vector(1,1){1}}
    %plot decision region
    %\put(2,-8){\line(-1,-2){1}}
    %\put(1,-7){\line(-1,-2){1}}
    %\put(0.5,-6.5){\line(-1,-2){.5}}
    %\put(1.5,-7.5){\line(-1,-2){1.25}}
    %\put(2.5,-9){\line(-1,-2){.5}}

    %plot maximizer
    \color{green}
    \put(-1,-8){\line(1,-1){4}}
\end{picture}
\end{document}