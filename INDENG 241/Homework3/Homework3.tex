\documentclass{article}
\usepackage{array}
\usepackage{amsfonts}
\usepackage{amsmath}
\usepackage{geometry}
\usepackage{stmaryrd}
\geometry{hmargin=2.5cm,vmargin=1.5cm}


\title{Homework 3}
\author{Arnaud Minondo}

\begin{document}
\maketitle
\section*{Exercise 1}
\subsection*{1.1}
Suppose $E\searrow F$  ie. $\mathbb{P}(F|E)\leq \mathbb{P}(F)$ then $\frac{\mathbb{P}(F\cap E)}{\mathbb{P}(E)}\leq \mathbb{P}(F)$ so $\frac{\mathbb{P}(F \cap E)}{\mathbb{P}(F)}\leq \mathbb{P}(E)$ and $ \mathbb{P}(E|F)\leq\mathbb{P}(E)$ so $F\searrow E$.
\subsection*{1.2}
This proposition is false : Consider rolling a fair dice and observing the result, E :``Obtain an odd number'', F:``Obtain a 6'', G:``Obtain an even number''.
\\
$\mathbb{P}(E|F) = 0 \leq \mathbb{P}(E)$ so $F\searrow E$. Moreover $\mathbb{P}(G|E) = 0 \leq \mathbb{P}(G)$ so $E\searrow G$. But $\mathbb{P}(G|F) = 1$ and $\mathbb{P}(G)=\frac{1}{2}$, the proposition is false.
\subsection*{1.3}
This proposition is also false : 
Consider rolling a fair dice and observing the result.
 Define E:``the dice fall on either 1 or 6'', F:``the dice fall on 1 or 2 or 3'' and G:``the dice fall on 1 or 4 or 5''.
\\
 $\mathbb{P}(E) = \frac{2}{6} = \frac{1}{3}$, $\mathbb{P}(E|G) = \frac{1}{3}$ so $G\searrow E$. Moreover, $\mathbb{P}(E|F) = \frac{1}{3}$ so $F\searrow E$.
\\
But $\mathbb{P}(E|F\cap G) = 1\ge \mathbb{P}(E)$ so the proposition is false.
\section*{Exercise 2}
\subsection*{2.1}
Define two events : A:``machine M1 does not work'' and B:``machine M2 does not work''.
\\
The probability that no machine work is : $\mathbb{P}(A\cap B) = \mathbb{P}(B|A)\mathbb{P}(A) = 0.4*0.01 = 0.004$.
\\
\subsection*{2.2}
The probability that at least one machine work is : $\mathbb{P}(\overline{A}\cup\overline{B}) = 1 - \mathbb{P}(A\cap B) = 0.996$.
\section*{Exercise 3}
Event A happens with probability $\frac{1}{2}$. Event B happens with probability $(\frac{4}{5})^3=0.512$. Finally event C happens with probability $(\frac{9}{10})^7 = 0.48$.
\\
B is the most likely to happen. The second most likely to happen is A and the less likely to happen is C.
\section*{Exercise 4}
\subsection*{4.1}
Define the event : $D_n$ as ``the day n is dry'', $P_n=\mathbb{P}(D_n)$.
\\
Then $\forall n \in \mathbb{N}, D_n\cup\overline{D_n} = \Omega$ with a disjoint union. \\
Thus $P_n = \mathbb{P}(D_n|D_{n-1})\mathbb{P}(D_{n-1})+\mathbb{P}(D_n|\overline{D_{n-1}})\mathbb{P}(\overline{D_{n-1}}) = pP_{n-1}+(1-p)(1-P_{n-1}) = (2p-1)P_{n-1}+(1-p)$
\subsection*{4.2}
$u_n = P_n-\frac{1}{2} = (2p-1)P_{n-1}+(1-p)-\frac{1}{2} = (2p-1)(P_{n-1}-\frac{1}{2}+\frac{1}{2})+\frac{1}{2}-p$ 
\\
$= (2p-1)u_{n-1} + \frac{1}{2}(2p -1)+\frac{1}{2}-p = (2p-1)u_{n-1}$
\subsection*{4.3}
As $u_n = (2p-1)u_{n-1}$, $u$ is a geometric sequence so $u_n = (2p-1)^n u_0 = \frac{1}{2}(2p-1)^n$\\
So $P_n=u_n +\frac{1}{2} = \frac{1}{2}((2p-1)^n+1)$
\section*{Exercise 5}
\subsection*{5.1}
Define the Event $A$:``a coupon of each type is chosen". As a coupon of each type is
 chosen and there is as much coupon chosen than different types of coupon, there exist a bijection
 between the index of the coupon and the type of the coupon. It means that there is $n!$ possibilities for chosing a coupon of
 each type. \\
 As there each choose has a probability of $\Pi_{i=1}^n p_i$:  $$\boxed{ \mathbb{P}(A) = n!(\Pi_{i=1}^n p_i) }$$

\subsection*{5.2}
Define $E_i$ : ``no coupons of type i is chosen". The Event $A$ : ``a coupon of each type is chosen"$=\cap_{i=1}^n\overline{E_i}$.
\\
So : $\mathbb{P}(\cup_{i=1}^nE_i) = 1 - \mathbb{P}(\cap_{i=1}^n\overline{E_i}) =1- n!(\Pi_{i=1}^n p_i) = \frac{n^n-n!}{n^n}$ with $p_i = \frac{1}{n}$
\\\\
Moreover : $\mathbb{P}(\cup_{i=1}^nE_i) = \sum_{k=1}^n(-1)^{k+1}\sum_{i_1<i_2<...<i_k}\mathbb{P}(\cap_{r=1}^kE_{i_r})$ after the inclusion exclusion principle.
\\\\
$\forall k\in\{1,2,...,n\}, \forall(i_1,i_2,...,i_k)\in\{1,2,...,n\}^k | i_1<i_2<...<i_k, \mathbb{P}(\cap_{r=1}^kE_{i_r}) = P_{n,k}$
 because each coupon can be exchanged with another one, the index type of the coupon does not matter as long as you don't pick 
 k coupons in the n available. So $\forall n \in \mathbb{N}, \forall k\in\{1,2,...,n\}, P_{n,k} = \frac{(n-k)^n}{n^n}$
\\\\
And : 
$\forall k\in\{1,2,...,n\}, \forall(i_1,i_2,...,i_k)\in\{1,2,...,n\}^k, \sum_{i_1<i_2<...<i_k} 1 = \binom{n}{k}$ 
 as each choice of k integer in $\{1,2,...,n\}$ is a good choice for $(i_1,i_2,...,i_k)$ assigning the lowest to $i_1$, the second lower to $i_2$
 and so on till $i_k$ which is the biggest of the selection.
\\\\
With those two results : $\mathbb{P}(\cup_{i=1}^nE_i) = \sum_{k=1}^n(-1)^{k+1}\sum_{i_1<i_2<...<i_k}\mathbb{P}(\cap_{r=1}^kE_{i_r}) = \sum_{k=1}^n (-1)^{k+1}\binom{n}{k}\frac{(n-k)^n}{n^n}$
\\\\
We have obtained that : $\frac{n^n-n!}{n^n} = \sum_{k=1}^n (-1)^{k+1}\binom{n}{k}\frac{(n-k)^n}{n^n}$.
 We can simplify multiplying both sides by $n^n$ and rearranging each term on the good side we obtain :
 $n!= n^n-\sum_{k=1}^n (-1)^{k+1}\binom{n}{k}(n-k)^n = n^n+\sum_{k=1}^n (-1)^k\binom{n}{k}(n-k)^n$ noticing that $n^n$ is the term for $k=0$ of the sum :
 $$\boxed{n! = \sum_{k=0}^n(-1)^k\binom{n}{k}(n-k)^n}$$ 
\section*{Exercise 6}
\subsection*{6.1}
Define $H_n$ : ``No consecutive 3 heads appear in n tosses of a faire coin'' ; and $t_n$ :``the toss n result in head''.
\\
We can decompose : $H_n = (H_{n-3}\cap(\overline{t_{n-2}},t_{n-1},t_{n}))\cup(H_{n-2}\cap(\overline{t_{n-1}},t_n))\cup(H_{n-1}\cap\overline{t_n})$
Which is a disjoint union.
\\
Moreover $\forall i \in \mathbb{N}, \forall j\in\mathbb{N}, j > i :  H_i$ is independent of $t_j $.
\\
So : $q_n=\mathbb{P}(H_n) = \mathbb{P}(H_{n-3})\mathbb{P}(\overline{t_{n-2}},t_{n-1},t_{n})+\mathbb{P}(H_{n-2})\mathbb{P}(\overline{t_{n-1}},t_n)+\mathbb{P}(H_{n-1})\mathbb{P}(\overline{t_n})$
\\
We obtain the final formula : $$q_n = \frac{1}{8}q_{n-3}+\frac{1}{4}q_{n-2}+\frac{1}{2}q_{n-1}$$
\subsection*{6.2}
$q_{10} = 0.49$
\end{document}