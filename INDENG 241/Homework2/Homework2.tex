\documentclass{article}
\usepackage{array}
\usepackage{amsfonts}
\usepackage{amsmath}
\usepackage{geometry}
\usepackage{stmaryrd}
\geometry{hmargin=2.5cm,vmargin=1.5cm}


\title{Homework 2}
\author{Arnaud Minondo}

\begin{document}
    \maketitle

    \section*{Exercise 1}
    If $\mathbb{P}(A|B) = 1$ then it means $B\subset A $ so $A^c\subset B^c$ and finally $\mathbb{P}(B^c|A^c) = 1$
    \section*{Exercise 2}
    I will denote A :``the two dice fall on different number'' and B : ``One of the two dice fall on 6'' 
    \\
    For A : there are 6 possibilities for the first and 5 for the second.
    \\
    Thus there are $6*5$ possibilities.
    \\
    B is the contrary of no any of the dice fall on 6 there is 5 possibilities for the first dice and 4 for the second.
    \\
    Thus there are $6*5-5*4 = 10$ possibilities.
    \\
    And $\mathbb{P}(B|A) = \frac{10}{30} = \frac{1}{3}$
    \section*{Exercise 3}
    The proportion of person having voted is $P = 0.35*0.46 + 0.62*0.30+0.24*0.58 = 0.4862$
    There is a proportion of : $I = 0.35*0.46 = 0.161$ independents, $C = 0.58*0.24$ conservatives and $L = 0.62*0.30$ liberals.
    \subsection*{3.1}
    $\mathbb{P}(\text{``The person is an Independent''}|\text{``Has voted''}) = \frac{I}{P} = \frac{0.1610}{0.4862} = 0,33$.
    \subsection*{3.2}
    $\mathbb{P}(\text{``The person is a Liberal''}|\text{``Has voted''}) = \frac{L}{P} = \frac{0.186}{0.4862} = 0,38$.
    \subsection*{3.3}
    $\mathbb{P}(\text{``The person is an Conservative''}|\text{``Has voted''}) = \frac{I}{P} = \frac{0.1392}{0.4862} = 0,29$.
    \subsection*{3.4}
    The fraction of voters that participated in the local election is : $P = 0.49$
    \section*{Exercise 4}
    \subsection*{4.1}
    We can deduce that both parents have the pair of genes : (Blue, Brown).
    \\
    There are 3 possibilities for the genes of Smith : (Blue, Brown), (Brown, Blue), (Brown, Brown)
    \\
    Only two of them include a blue-eyed gene so the probability is :$\frac{2}{3}$
    \subsection*{4.2}
    I will denote $C_1$ : ``The first Child has brown eyes'' and $S_{ij}$ : ``Smith has the pair of genes $(i,j)$'' ; B will denote the Brown gene and b the blue one.
    \\
    The fact that the mother has blue eyes means she has the pair of genes : bb
    \\
    $\mathbb{P}(\bar{C_1}) = \mathbb{P}(\bar{C_1}|S_{BB})\mathbb{P}(S_{BB})+\mathbb{P}(\bar{C_1}|S_{Bb})\mathbb{P}(S_{Bb})$
    \\
    In the case where Smith is BB : his child can't have blue eyes as he will at least have one gene B from Smith. So $\mathbb{P}(C_1|S_{BB}) = 0$.
    \\
    In the case where Smith is Bb : then there are only two cases : Smith gives b to his child which will result in blue eyes ; or Smith give his B and the child will have brown eyes.
    Only one on the two cases will result in $C_1$ so $\mathbb{P}(\bar{C_1}|S_{Bb}) = \frac{1}{2}$.
    \\
    So $\mathbb{P}(\bar{C_1}) = 0 + \frac{1}{2}\frac{2}{3} = \frac{1}{3}$
    \subsection*{4.3}
    I will denote $C_2$ : ``Their second child has brown eyes''.
    \\
    $\mathbb{P}(C_2|C_1) = \frac{\mathbb{P}(C_2\cap C_1)}{\mathbb{P}(C_1)}$
    \\
    $\mathbb{P}(C_2\cap C_1) = \mathbb{P}(C_2 \cap C_1 \cap S_{Bb}) + \mathbb{P}(C_2 \cap C_1 \cap S_{BB}) = \mathbb{P}(C_2 \cap C_1 | S_{Bb})\mathbb{P}(S_{Bb}) +\mathbb{P}(C_2 \cap C_1 | S_{BB})\mathbb{P}(S_{BB})$
    \\
    $ = \frac{1}{2^2}\frac{2}{3}+ 1*\frac{1}{3} = \frac{1}{2}$
    \\
    So finally : $\mathbb{P}(C_2|C_1) = \frac{\frac{1}{2}}{\frac{2}{3}} = \frac{3}{4}$
    \section*{Exercise 5}
    \subsection*{5.1}
    I will denote $G_{ij}$ the event : the child is (i,j).
    For the child there is 2 possibilities : either $G_{AA}$, $G_{Aa}$, if we don't care about the order.
    \\
    Define B : ``The off-spring is an albino''.
    \\
    $\mathbb{P}(B) = \mathbb{P}(B|G_{AA})\mathbb{P}(G_{AA}) + \mathbb{P}(B|G_{Aa})\mathbb{P}(G_{Aa})$\\
    $=0*\frac{1}{3}+\frac{1}{4}\frac{2}{3} = \frac{1}{6}$    
    \subsection*{5.2}
    Define C : ``The second offspring is an albino''
    \\
    $\mathbb{P}(C|\bar{B}) = \frac{\mathbb{P}(C\cap \bar{B})}{\mathbb{P}(\bar{B})}$
    \\
    We already have $\mathbb{P}(B) = \frac{5}{6}$ then we just have to derive $\mathbb{P}(C\cap \bar{B})$
    $\mathbb{P}(C \cap \bar{B}) = \mathbb{P}(C\cap \bar{B}\cap G_{AA})+\mathbb{P}(C\cap \bar{B}\cap G_{Aa}) = 0+\frac{1}{4}\frac{3}{4}\frac{2}{3} = \frac{1}{8} $
    Then the probability that the second child is an albino is  : $\frac{\frac{1}{8}}{\frac{5}{6}} = \frac{3}{20}$
    
    \section*{Exercise 6}
    \subsection*{6.1}
    I will denote :``The i-th child is a boy'' as $B_i$ and  ``All children are of the same sex'' as $S$.
    \\
    So let X be the number of boy, then X follows a binomial distribution : $\mathcal{B}(5,\frac{1}{2})$.
    \\
    $\mathbb{P}(S) = \mathbb{P}((\cap_{i=1}^5 B_i)\cup(\cap_{i=1}^5 B_i)) = (\frac{1}{2})^5+(\frac{1}{2})^5 = \frac{1}{16}$
    \subsection*{6.2}
    I will denote D :`` The 3 eldest are boy and the other girls''.
    \\
    $\mathbb{P}(D) = \mathbb{P}(\bar{B_1}\bar{B_2}B_3B_4B_5) = (\frac{1}{2})^5 = \frac{1}{32}$
    \subsection*{6.3}
    B :``Exactly 3 are boys'', the result is given by : $\mathbb{P}(X = 3)$ where X follows a binomial distribution.
    \\
    $\mathbb{P}(B) = \binom{5}{3}(\frac{1}{2})^5$
    \subsection*{6.4}
    O : ``The  2 oldest are girls''
    \\
    $\mathbb{P}(O) = \mathbb{P}((B_1\cup\bar{B_1})(B_2\cup\bar{B_2})(B_3\cup\bar{B_3})\bar{B_4}\bar{B_5}) = \mathbb{P}(\bar{B_4})\mathbb{P}(\bar{B_5})=\frac{1}{4}$
    \subsection*{6.5}
    G : ``There is at least 1 girl''
    \\
    $\mathbb{P}(G) = 1 - \mathbb{P}(\bar{G})=1 - \mathbb{P}(B_1B_2B_3B_4B_5) = 1 - (\frac{1}{2})^5  = \frac{31}{32}$
\end{document}