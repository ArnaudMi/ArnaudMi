\documentclass{article}

\usepackage{array}
\usepackage{amsfonts}
\usepackage{amsmath}
\usepackage{geometry}
\usepackage{stmaryrd}

\geometry{hmargin=2.5cm,vmargin=1.5cm}
\title{IEOR 241 : Homework 6}
\author{Arnaud Minondo}

\begin{document}
\maketitle
\section*{Exercise 1 :}
\subsection*{(a)}
$\int_{-1}^{1}c(1-x^2)dx = c(2 - \frac{2}{3}) = c\frac{4}{3}$ so $c = \frac{3}{4} $.
\subsection*{(b)}
Let $t\in[-1;1]$ : $\mathbb{P}(X\leq t)=\int_{-1}^{t}\frac{3}{4}(1-x^2)dx = \frac{3}{4}(t+1 - \frac{t^3}{3}-\frac{1}{3}) = \frac{1}{2}+\frac{3}{4}t-\frac{t^3}{4}$.
\section*{Exercise 2 :}
Let $S$ denote the sales of the filling station.\\
Let $t\in[0;1]$, $\mathbb{P}(S\leq t) = \int_{0}^t f(x)dx = \int_{0}^t 5(1-x)^4dx = -(1-t)^5+1$.
\\
Let $c$ be the capacity of the filling station. $\mathbb{P}(S\ge c) = 1- \mathbb{P}(S\leq c) = 0.01$
\\
Thus : $1-1+(1-c)^5 = 0.01$ which yields : $$\boxed{c = 1 - (0.01)^\frac{1}{5} = 0.60}$$
\section*{Exercise 3 :}
Let P be the position of the point in the line of length L : $P\sim\mathcal{U}([0;L])$ a uniform law.
\\
The ratio of the smaller over the larger is : $Z = \frac{\min(U,L-U)}{\max(U,L-U)}$.
\\\\
$\mathbb{P}(Z\leq \frac{1}{4}) = \mathbb{P}(Z\leq \frac{1}{4}| U\leq \frac{1}{2})\mathbb{P}(U\leq \frac{1}{2})+\mathbb{P}(Z\leq \frac{1}{4}|U> \frac{1}{2})\mathbb{P}(U> \frac{1}{2})$
\\\\
As $U$ and $L-U$ follow the same law they play simetric role and $\mathbb{P}(\frac{U}{L-U}\leq\frac{1}{4}) = \mathbb{P}(\frac{L-U}{U}\leq \frac{1}{4})$
\\
$\mathbb{P}(Z\leq \frac{1}{4}) = \mathbb{P}(\frac{U}{L-U}\leq \frac{1}{4})\frac{1}{2}+ \mathbb{P}(\frac{L-U}{U}\leq \frac{1}{4})\frac{1}{2} = 2\mathbb{P}(\frac{5}{4}U\leq \frac{L}{4}) = 2\mathbb{P}(U\leq \frac{L}{5}) = 2\frac{L}{5}\frac{1}{L} = \frac{2}{5}$
\section*{Exercise 4 :}
$\mathbb{P}(X>5) = \mathbb{P}(Z>-\frac{5}{6}) \simeq 0.79 $
\\
$\mathbb{P}(X<8) = \mathbb{P}(Z<-\frac{1}{3}) \simeq 0.30 $
\\
$\mathbb{P}(X>16) = \mathbb{P}(Z>1) \simeq 0.16$
\section*{Exercise 5 :}
Let $X$ be the number of left handed in the school. $X\sim\mathcal{B}(200,\frac{1}{5})$
\\
The probability that there is at least 20 left handed is : $\mathbb{P}(X \ge 20) = \sum\limits_{i=20}^{200} \binom{200}{i}(\frac{1}{5})^i(\frac{4}{5})^{200-i} = 0.99$
\\
You can approximate the law of X by a $\mathcal{N}(40,32)$ normal distribution. The approximation gives $\mathbb{P}(X>20) \simeq \mathbb{P}(Z>-3.53) = 0.99$
\section*{Exercise 6 :}
\subsection*{(a)}
Let $T$ be the repair time, $T\sim\mathcal{E}(\frac{1}{2})$.
\\
$\mathbb{P}(T>2) = \exp(-\frac{1}{2}2) = \exp(-1)\simeq 0.37$
\subsection*{(b)}
$\mathbb{P}(T>10|T>9) = \mathbb{P}(T>1) = \exp(-\frac{1}{2})\simeq 0.61$ because the exponantial law is memoryless.
\section*{Exercise 7 :}
Let $T$ the function time of the bought radio.
$\mathbb{P}(T>8) = \exp(-\frac{1}{8}8) = \exp(-1)\simeq 0.37$
\end{document}